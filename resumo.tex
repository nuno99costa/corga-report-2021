%!TEX root = main.tex

\section*{Resumo}
\addcontentsline{toc}{section}{Resumo}

O presente relatório apresenta a organização Agência Continental, uma análise da mesma e uma descrição do processo de obtenção de dados, tudo no âmbito da Unidade Curricular de CORGA (Comportamento organizacional).

Foram distribuídos inquéritos, com base no modelo OCAI aos colaboradores da organização de modo a obter dados relativos à opinião destes sobre a cultura organizacional em que se encontram inseridos.

Após uma análise aos inquéritos, verificamos que, na sua maioria, os funcionários têm uma opinião positiva sobre a cultura organizacional da empresa. No entanto, existem sempre pontos fracos (ou menos positivos), que são também demonstrados na análise dos dados obtidos.

O processo de desenvolvimento deste trabalho foi uma experiência positiva na vertente profissional dos membros do grupo, na medida que nos permitiu adquirir uma nova visão no que toca à recolha e análise de dados num ambiente empresarial, para além da verificação das implicações da cultura organizacional no bem estar dos colaboradores e, subsequentemente, no sucesso da organização.

\subsection*{Palavras-chave:} 

Organização, Cultura Organizacional, Inquéritos, Análise de resultados, OCAI